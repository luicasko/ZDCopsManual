\chapter{Review of Zero Degree Calorimeter (ZDC) electronics and cabling}
\section{ZDC Photomultipiers}
Currently, the installed photomultiplier assembly (photomultiplier tube, voltage-divider circuit and other components, all integrated into a single case) is Hamamatsu H2431-50~\cite{PMTtube}, currently using the default the photomultiplier tube (PMT) Hamamatsu R2083 (this differs from the ZDC design report \cite{ZDCdocumentation}).

\section{High-voltage power supply}

\begin{figure}[htb]
\begin{center}
\includegraphics[width=.7\textwidth]{img/hvsupplies.jpg}
\end{center}
\caption{Power supplies, that are used for ZDC PMTs.}
\label{hvsupplies}
\end{figure}

The ZDC PMTs are connected to the high-voltage (HV) crate LeCroy 1440, situated on the STAR support platform next the STAR detector--barrel. In the LeCroy crate, there are two power supply cards LeCroy 1444N that are connected to the low-voltage (LV) supply cards Lecroy 1441 (Fig.\ref{hvsupplies}). Generally, this assembly has a maximum HV output of 5 kV, but on the cables connected to PMTs, there is a component, that limits maximum output to 3 kV. This is for the protection of the PMTs which have a maximum oprating voltage up to 3$\,$kV \cite{PMTtube}.

\begin{figure}[htb]
\begin{center}
\includegraphics[width=.7\textwidth]{img/pp1.jpg}
\end{center}
\caption{Patch panel 1 for ZDC west.  ZDC cables are on the left, high voltage are red and output from PMTS are black.}
\label{pp1}
\end{figure}

\begin{figure}[htb]
\begin{center}
\includegraphics[width=.7\textwidth]{img/pp2.jpg}
\end{center}
\caption{Patch panel 2 for ZDC east high voltage (red cables) and output from PMTS (black cables).}
\label{pp2}
\end{figure}

High voltage cables are not directly connected to the PMTs, they are connected thrrough patch panels PP1 for the west and PP2 for the east (see Fig.~\ref{pp1} and Fig.~\ref{pp2}). Both of the patch pannels are located on the sides of the STAR infrastructer platform. Output from PMTs is connected to the logic units via PP1 and PP2 as well.

Power supplies are accessed via an RS232 port that is connected to the, so called, BDB computer. On this computer, there are IOC (input/output controlers) for several detectors. The one for ZDC can be accessed from the \texttt{sc5.starp.bnl.gov} computer (SC5).

\subsection{HV power supply cable map}
\begin{table}[htb] 
\caption{\label{HVtable}Demand voltages of the ZDC tower--channels and their positions in the LeCroy crate for Run16 and Run17. The letters in the channel positions are the LeCroy channel representation in the slow controls.}
\label{corected}
\begin{center}
\begin{tabular}{lcccccc}
\toprule
\multicolumn{3}{c}{} &  \multicolumn{2}{c}{Run16} & \multicolumn{2}{c}{Run17} \\
 &Tower&Voltage [V]  &  Slot&Channel  &  Slot&Channel\\
\midrule
East  &1 & 2540 &9&1F&  9&1F\\
      &2 & 3000 &8&3H&  7&3H\\
      &3 & 2575 &9&2G&  9&2G\\
\midrule
West  &1 & 2558 &9&3H&  9&3H \\
      &2 & 2748 &8&2G&  7&2G \\
      &3 & 3000 &9&5J&  7&5J \\
\bottomrule
\end{tabular}
\end{center}
\end{table}



\subsection{Past HV power supply issues}
In 2017, the channels did not ramp all the way to the desired values (see Table ???), but stayed on $\sim$2500 V\@. As was found out, the problematic part was an old LeCroy 1441 low-voltage power supply card in the front of the LeCroy crate which did not handle the power throughput. This issue was solved by connecting the HV channels with a demand voltage above $\sim$2600 V to another slot on a different
LeCroy-1441 supply.

\section{Signal output}

Each PMT has its own signal output. All of them go through PP1 or PP2 to their counterparts, that are close to TCIM unit (Fig.~\ref{tcimpp}). Then, they are connected to logic unit in the, so called, NIM crate (Fig.~\ref{lu_sum}). Two of the outputs from this unit are sum signals, one for ZDC east and the other for ZDC west. Another output is "ZDC and" signal, which is the coincidence signal of the east and the west sums. All of these, together with signal from every PMT, are connected to TCIM.

\begin{figure}[htb]
\begin{center}
\includegraphics[width=.7\textwidth]{img/tcimpp.jpg}
\end{center}
\caption{TCIM and patch panel for the ZDC east.}
\label{tcimpp}
\end{figure}


\begin{figure}[htb]
\begin{center}
\includegraphics[width=.7\textwidth]{img/lusum.jpg}
\end{center}
\caption{Logic unit for ZDC signal processing.}
\label{lu_sum}
\end{figure}

The TCIM is another logic unit.  It can be accessed from any  computer within the STAR network
from a browser on the website\\ 
\url{172.16.15.101/AnalogandCoincidenceLogic13.html}.\\
The power switch of the TCIM is at \texttt{tof@tofrnps} (tofrpns = TOF remote network power switch).
Using TCIM, ZDC signal can be controlled. Here, the ``ZDC kill'' signal is connected as well. This is a TTL signal that nullifies (kills) everything going from the ZDC in preset time intervals. The
kill signal frequency can be changed from the TCIM website.

TTL output from TCIM is connected to another logic unit, situated in DAQ room at STAR. Rescalers, calculated with this unit, can be accessed from SC5 computer.

TDC (timing of the ZDC events) can be set at \texttt{startrg.starp.bnl.gov}. At this computer, status and control of TCIM can be accessed via a browser.

\chapter{ZDC remote control}
\section{\label{computers}List of important computers and commands that control the ZDC electronics}

To gain access all of the important detector control hosts, one needs to add request at \url{https://www.star.bnl.gov/starkeyw/}. All of the hosts listed below, can be accessed from RCF using RSA public key of your computer, if it is not described otherwise.
\begin{outline}
 \1 \texttt{stargw.starp.bnl.gov}, account name: STAR user name
   \2 STAR Gateways
 \1 \texttt{sysuser@sc5.starp.bnl.gov} 
   \2 operation and control of the ZDC power supplies
   \2 commands: \texttt{bbchv}, \texttt{zdc}, \texttt{richscal}
 \1 \texttt{staruser@startrg.starp.bnl.gov}
   \2 TDC timing of ZDC
 \1 \texttt{sysuser@alh.starp.bnl.gov} 
   \2 old alarm handler for slow controls at STAR
 \1 \texttt{sysuser@alh2.starp.bnl.gov}
   \2 new alarm handler for slow controls at STAR
 \1 \texttt{tof@tofrnps}
   \2 TOF remote network power switch
   \2 accessible only from STAR Gateway
   \2 login password can be found in the "password sheet" for shift leaders
   \2 operation of the power switch for TCIM (especially useful when rebooting the TCIM)
 \1 \url{172.16.15.101/AnalogandCoincidenceLogic13.html}
   \2 accessible from \texttt{startrg.starp.bnl.gov}
   \2 Webpage with the TCIM coincidence logic (it is bookmarked e.g.\ in Firefox on \texttt{startrg.starp.bnl.gov}).
 \1 \texttt{telnet scserv 9010}: 
   \2 accessible  from \texttt{sc5.starp.bnl.gov}
   \2 Can be used for communication with the LeCroy 1440 crate (ZDC HV power supply)
   \2 BDB computer
   \2 to restart the ZDC IOC on BDB, see section \ref{bdbreboot}.
\end{outline}


\section{Power supplies control}
Power supplies can be accessed and controlled using graphical windows, that can be opened at SC5 computer. 
Window for BBC HV (Fig.~\ref{bbchv}) is opened with command \texttt{bbchv}. From this windows, other control windows can be accessed, showing actual performance of ZDC, SMD and VPD HV. AC trips can be controlled and cleared as well. Green dots on the right panel shows if the HV supply is fully on or off. Trips can be cleared only if HV is on. ZDC, upVPD and SMD can be turned on and off from here, with the buttons on the left side.

Not all of the control buttons on the BBC HV window trigger opening of another graphical window or any direct response. However, they work as a command, which results in showing of some data in terminal at BDB computer.

 
\begin{figure}[htb]
  \begin{center}
    \includegraphics[width=1.\textwidth]{img/bbchv.png}
  \end{center}
  \caption{Graphical window for control of high voltage power supplies, opened after entering \texttt{bbchv} command at SC5 computer.}
\label{bbchv}
\end{figure}

A detailed ZDC HV control window (Fig.~\ref{zdcwindow}) can be opened either by clicking on the button with window icon, next to the title "ZDC" in the left panel of BBC HV control window, or by typing \texttt{zdc} command in terminal at SC5 computer. Set values of HV on all of the PMTs and their readback values are shown in this window.  ZDC can be turned on and off also from here.

\begin{figure}[htb]
  \begin{center}
    \includegraphics[width=.6\textwidth]{img/zdc.png}
  \end{center}
  \caption{Graphical window for control of high voltage power supplies, opened after entering \texttt{zdc} command at SC5 computer.}
\label{zdcwindow}
\end{figure}

Values of RICH Scaler Rates for BBC and ZDC detectors can be displayed with \texttt{richscal} command at SC5 computer. "ZDC-nokill" is the event rate without "ZDC-kill" signal. As it was already mentioned, "ZDC And" is the rate of ZDC east and ZDC west coincidence.

\begin{figure}[htb]
  \begin{center}
    \includegraphics[width=.6\textwidth]{img/richscal.png}
  \end{center}
  \caption{Graphical window for control of high voltage power supplies, opened after entering \texttt{richscal} command at SC5 computer.}
\label{richscal}
\end{figure}

% \newpage

\subsection{Notes}
On some computers, when opening the Slow Controls windows (by Epics) they cause a Segmentation Violation and refuse to open. This issue is caused by the scalable fonts that need to be installed on your computer, in order to open graphical applications at remote hosts.
You can solve this issue by installing the package \texttt{xfonts-100dpi}, to install it on e.g.\ Debian based systems, just type in the terminal on your computer:
\begin{verbatim}
$ sudo apt-get install xfonts-100dpi
\end{verbatim}

\subsection{Instructions for the BDB reboot}
\label{bdbreboot}

\begin{enumerate}
\item Connect or go to \texttt{sc5.starp.bnl.gov} computer.
\item In a termnial, type \texttt{telnet scserv 9010}, and hit \texttt{[Enter]}.
\item Hit  \texttt{[Enter]} again.
\item If upVPD is on, turn it off.
\item Press \texttt{Ctrl + X} to reboot the processor.
\item Wait until the processors reboot information comes to the following lines:
\noindent
\begin{verbatim}
 Done executing startup script
 /home/sysuser/epics3.13.10/Application/iocBoot/iocbbc/stold.cmd
 bdb> checking system
\end{verbatim}

\begin{itemize}
 \item Please note that sometimes you might need to scroll up in case you missed this line appearing in the display.
 \item During the process the following GUIs will go white: upVPD, BBC LeCroy and ZDC. Ignore any messages you see on the Linux screens afterwards. If a GUI remains blank but the reset is complete, try to maximize and restore the GUI to their original size.
\end{itemize}

\item When the terminal is done updating and all GUIs are back, \textbf{turn on what needs to be on}: BBC HV, ZDC HV, EAST/WEST SMD and PP2PP HV. \textbf{Check the required detector state for the current conditions.}
\item Press \texttt{Ctrl + ]}. This should bring the prompt to \texttt{telnet>}.
\item Type \texttt{quit} and hit \texttt{[Enter]}.
\item Make an entry in the shift log when done.
\end{enumerate}

These instructions are (and should be) also placed in the STAR control room.

\subsection{Instructions for the TCIM reboot}
Sometimes, the ZDC RICH scalers suddenly change to weird values (like 200k for the ``ZDC and''). This can be caused by a fault in logic on the TCIM board and can be solved by rebooting it.
To do that, follow these steps:


\begin{enumerate}
\item Type into a terminal somewhere in the STAR network (e.g.\ from the \texttt{stargw} -- see section~\ref{computers})
\begin{verbatim}
ssh tof@tofrnps
\end{verbatim}
\item you will be asked for a password (it is in the password sheet)
\item enter
\item \verb=/off B4, y=
\item this turns off the TCIM. Wait 20--30 seconds
\item \verb=/on B4, y=
\item \verb=/x=
\end{enumerate}

\chapter{Tests of the hardware before the start of the collisions at RHIC}
The PMTs' response to cosmics needs to be checked before the run. After you turn the ZDC HV on, you can check the signal with an oscilloscope on the signal output at the patch panels close to the TCIM (See Fig.\ \ref{tcimpp}). A good way of doing this is to turn on the the HV on the towers one-by-one so you can see if the output of some ZDC towers is swapped.
Moreover, the ZDC cables may be swapped with the SMD, disconnected, or faulty. Therefore, if the signal is missing on the patch panel, you should follow the cables to the PMTs and check if all the cables are in the right place. Equally, you should check the HV connection.

\begin{figure}[htb]
\begin{center}
\includegraphics[width=.7\textwidth]{img/zdcepatch.jpg}
\end{center}
\caption{Outputs from PMTs at patch panel for ZDC east, that needs to be checked.}
\label{zdcepatch}
\end{figure}

ZDC west and east sum signals are typically checked on outputs of both TCIM and logic unit. ``ZDC and'' and ``ZDC kill'' signals are controlled on TCIM using  signal from a pulser. The pulser needs to be connected to one of the channels of ZDC East and West simultaneously. Typically, you divide the signal from the pulser and connect it to the ZDC East1 and West1 towers input of the TCIM. Now, you should monitor the RICH scalers output GUI on the \texttt{sc5} computer and the TCIM coincidence logic webpage at \url{172.16.15.101/AnalogandCoincidenceLogic13.html}.
