\documentclass[a4paper,10pt]{report}
\usepackage[utf8]{inputenc}
\usepackage{amsmath}
\usepackage{amsfonts}
\usepackage{breakurl}
\usepackage{indentfirst}
\usepackage{graphicx}
\usepackage{enumerate}
\usepackage[breaklinks, hidelinks]{hyperref}
\usepackage{url}
\usepackage{booktabs}

\newcommand{\vect}[1]{\ensuremath{\boldsymbol{#1}}}
\newcommand{\ddd}{\ensuremath{\,\mathrm{d}}}
\newcommand{\dd}{\mathrm{d}}
\newcommand{\Var}{\mathrm{Var}}
\newcommand{\pravYi}{P(Y_i|\alpha, \beta, \epsilon_i)}
\newcommand{\pravY}{P(\boldsymbol{Y}|\alpha, \beta, \epsilon_i)}
\newcommand{\nsigma}{\frac{N}{\sigma^2}}
\newcommand{\arpxsq}{\overline{(X^2)}}
\newcommand{\sigman}{\frac{\sigma^2}{N}}
\newcommand{\jmenovatel}{ \arpxsq - \overline{X}^2}
\usepackage{amsmath}


%opening
\title{ZDC Operator's Manual}
\author{Miroslav Šimko, Lukáš Kramárik}
\date{February 2017}

\begin{document}

\maketitle

\tableofcontents
\chapter{Review of Zero Degree Calorimeter (ZDC) electronics and cabling}
\section{ZDC Photomultipiers}
Currently, the installed photomultiplier assembly (photomultiplier tube, voltage-divider circuit and other components, all integrated into a single case) is Hamamatsu H2431-50~\cite{PMTtube}, currently using the default the photomultiplier tube (PMT) Hamamatsu R2083 (this differs from the ZDC design report \cite{ZDCdocumentation}).

\section{High-voltage power supply}

\begin{figure}[htb]
\begin{center}
\includegraphics[width=.7\textwidth]{img/hvsupplies.jpg}
\end{center}
\caption{Power supplies, that are used for ZDC PMTs.}
\label{hvsupplies}
\end{figure}

The ZDC PMTs are connected to the high-voltage (HV) crate LeCroy 1440, situated on the STAR support platform next the STAR detector--barrel. In the LeCroy crate, there are two power supply cards LeCroy 1444N that are connected to the low-voltage (LV) supply cards Lecroy 1441 (Fig.\ref{hvsupplies}). Generally, this assembly has a maximum HV output of 5 kV, but on the cables connected to PMTs, there is a component, that limits maximum output to 3 kV. This is for the protection of the PMTs which have a maximum oprating voltage up to 3$\,$kV \cite{PMTtube}.

\begin{figure}[htb]
\begin{center}
\includegraphics[width=.7\textwidth]{img/pp1.jpg}
\end{center}
\caption{Patch panel 1 for ZDC west.  ZDC cables are on the left, high voltage are red and output from PMTS are black.}
\label{pp1}
\end{figure}

\begin{figure}[htb]
\begin{center}
\includegraphics[width=.7\textwidth]{img/pp2.jpg}
\end{center}
\caption{Patch panel 2 for ZDC east high voltage (red cables) and output from PMTS (black cables).}
\label{pp2}
\end{figure}

High voltage cables are not directly connected to the PMTs, they are connected thrrough patch panels PP1 for the west and PP2 for the east (see Fig.~\ref{pp1} and Fig.~\ref{pp2}). Both of the patch pannels are located on the sides of the STAR infrastructer platform. Output from PMTs is connected to the logic units via PP1 and PP2 as well.

Power supplies are accessed via an RS232 port that is connected to the, so called, BDB computer. On this computer, there are IOC (input/output controlers) for several detectors. The one for ZDC can be accessed from the \texttt{sc5.starp.bnl.gov} computer (SC5).

\subsection{HV power supply cable map}
\begin{table}[htb] 
\caption{\label{HVtable}Demand voltages of the ZDC tower--channels and their positions in the LeCroy crate for Run16 and Run17. The letters in the channel positions are the LeCroy channel representation in the slow controls.}
\label{corected}
\begin{center}
\begin{tabular}{lcccccc}
\toprule
\multicolumn{3}{c}{} &  \multicolumn{2}{c}{Run16} & \multicolumn{2}{c}{Run17} \\
 &Tower&Voltage [V]  &  Slot&Channel  &  Slot&Channel\\
\midrule
East  &1 & 2540 &9&1F&  9&1F\\
      &2 & 3000 &8&3H&  7&3H\\
      &3 & 2575 &9&2G&  9&2G\\
\midrule
West  &1 & 2558 &9&3H&  9&3H \\
      &2 & 2748 &8&2G&  7&2G \\
      &3 & 3000 &9&5J&  7&5J \\
\bottomrule
\end{tabular}
\end{center}
\end{table}



\subsection{Past HV power supply issues}
In 2017, the channels did not ramp all the way to the desired values (see Table ???), but stayed on $\sim$2500 V\@. As was found out, the problematic part was an old LeCroy 1441 low-voltage power supply card in the front of the LeCroy crate which did not handle the power throughput. This issue was solved by connecting the HV channels with a demand voltage above $\sim$2600 V to another slot on a different
LeCroy-1441 supply.

\section{Signal output}

Each PMT has its own signal output. All of them go through PP1 or PP2 to their counterparts, that are close to TCIM unit (Fig.~\ref{tcimpp}). Then, they are connected to logic unit in the, so called, NIM crate (Fig.~\ref{lu_sum}). Two of the outputs from this unit are sum signals, one for ZDC east and the other for ZDC west. Another output is "ZDC and" signal, which is the coincidence signal of the east and the west sums. All of these, together with signal from every PMT, are connected to TCIM.

\begin{figure}[htb]
\begin{center}
\includegraphics[width=.7\textwidth]{img/tcimpp.jpg}
\end{center}
\caption{TCIM and patch panel for the ZDC east.}
\label{tcimpp}
\end{figure}


\begin{figure}[htb]
\begin{center}
\includegraphics[width=.7\textwidth]{img/lusum.jpg}
\end{center}
\caption{Logic unit for ZDC signal processing.}
\label{lu_sum}
\end{figure}

The TCIM is another logic unit.  It can be accessed from any  computer within the STAR network
from a browser on the website\\ 
\url{172.16.15.101/AnalogandCoincidenceLogic13.html}.\\
The power switch of the TCIM is at \texttt{tof@tofrnps} (tofrpns = TOF remote network power switch).
Using TCIM, ZDC signal can be controlled. Here, the ``ZDC kill'' signal is connected as well. This is a TTL signal that nullifies (kills) everything going from the ZDC in preset time intervals. The
kill signal frequency can be changed from the TCIM website.

TTL output from TCIM is connected to another logic unit, situated in DAQ room at STAR. Rescalers, calculated with this unit, can be accessed from SC5 computer.

TDC (timing of the ZDC events) can be set at \texttt{startrg.starp.bnl.gov}. At this computer, status and control of TCIM can be accessed via a browser.

\chapter{ZDC remote control}
\section{\label{computers}List of important computers and commands that control the ZDC electronics}

To gain access all of the important detector control hosts, one needs to add request at \url{https://www.star.bnl.gov/starkeyw/}. All of the hosts listed below, can be accessed from RCF using RSA public key of your computer, if it is not described otherwise.
\begin{outline}
 \1 \texttt{stargw.starp.bnl.gov}, account name: STAR user name
   \2 STAR Gateways
 \1 \texttt{sysuser@sc5.starp.bnl.gov} 
   \2 operation and control of the ZDC power supplies
   \2 commands: \texttt{bbchv}, \texttt{zdc}, \texttt{richscal}
 \1 \texttt{staruser@startrg.starp.bnl.gov}
   \2 TDC timing of ZDC
 \1 \texttt{sysuser@alh.starp.bnl.gov} 
   \2 old alarm handler for slow controls at STAR
 \1 \texttt{sysuser@alh2.starp.bnl.gov}
   \2 new alarm handler for slow controls at STAR
 \1 \texttt{tof@tofrnps}
   \2 TOF remote network power switch
   \2 accessible only from STAR Gateway
   \2 login password can be found in the "password sheet" for shift leaders
   \2 operation of the power switch for TCIM (especially useful when rebooting the TCIM)
 \1 \url{172.16.15.101/AnalogandCoincidenceLogic13.html}
   \2 accessible from \texttt{startrg.starp.bnl.gov}
   \2 Webpage with the TCIM coincidence logic (it is bookmarked e.g.\ in Firefox on \texttt{startrg.starp.bnl.gov}).
 \1 \texttt{telnet scserv 9010}: 
   \2 accessible  from \texttt{sc5.starp.bnl.gov}
   \2 Can be used for communication with the LeCroy 1440 crate (ZDC HV power supply)
   \2 BDB computer
   \2 to restart the ZDC IOC on BDB, see section \ref{bdbreboot}.
\end{outline}


\section{Power supplies control}
Power supplies can be accessed and controlled using graphical windows, that can be opened at SC5 computer. 
Window for BBC HV (Fig.~\ref{bbchv}) is opened with command \texttt{bbchv}. From this windows, other control windows can be accessed, showing actual performance of ZDC, SMD and VPD HV. AC trips can be controlled and cleared as well. Green dots on the right panel shows if the HV supply is fully on or off. Trips can be cleared only if HV is on. ZDC, upVPD and SMD can be turned on and off from here, with the buttons on the left side.

Not all of the control buttons on the BBC HV window trigger opening of another graphical window or any direct response. However, they work as a command, which results in showing of some data in terminal at BDB computer.

 
\begin{figure}[htb]
  \begin{center}
    \includegraphics[width=1.\textwidth]{img/bbchv.png}
  \end{center}
  \caption{Graphical window for control of high voltage power supplies, opened after entering \texttt{bbchv} command at SC5 computer.}
\label{bbchv}
\end{figure}

A detailed ZDC HV control window (Fig.~\ref{zdcwindow}) can be opened either by clicking on the button with window icon, next to the title "ZDC" in the left panel of BBC HV control window, or by typing \texttt{zdc} command in terminal at SC5 computer. Set values of HV on all of the PMTs and their readback values are shown in this window.  ZDC can be turned on and off also from here.

\begin{figure}[htb]
  \begin{center}
    \includegraphics[width=.6\textwidth]{img/zdc.png}
  \end{center}
  \caption{Graphical window for control of high voltage power supplies, opened after entering \texttt{zdc} command at SC5 computer.}
\label{zdcwindow}
\end{figure}

Values of RICH Scaler Rates for BBC and ZDC detectors can be displayed with \texttt{richscal} command at SC5 computer. "ZDC-nokill" is the event rate without "ZDC-kill" signal. As it was already mentioned, "ZDC And" is the rate of ZDC east and ZDC west coincidence.

\begin{figure}[htb]
  \begin{center}
    \includegraphics[width=.6\textwidth]{img/richscal.png}
  \end{center}
  \caption{Graphical window for control of high voltage power supplies, opened after entering \texttt{richscal} command at SC5 computer.}
\label{richscal}
\end{figure}

% \newpage

\subsection{Notes}
On some computers, when opening the Slow Controls windows (by Epics) they cause a Segmentation Violation and refuse to open. This issue is caused by the scalable fonts that need to be installed on your computer, in order to open graphical applications at remote hosts.
You can solve this issue by installing the package \texttt{xfonts-100dpi}, to install it on e.g.\ Debian based systems, just type in the terminal on your computer:
\begin{verbatim}
$ sudo apt-get install xfonts-100dpi
\end{verbatim}

\subsection{Instructions for the BDB reboot}
\label{bdbreboot}

\begin{enumerate}
\item Connect or go to \texttt{sc5.starp.bnl.gov} computer.
\item In a termnial, type \texttt{telnet scserv 9010}, and hit \texttt{[Enter]}.
\item Hit  \texttt{[Enter]} again.
\item If upVPD is on, turn it off.
\item Press \texttt{Ctrl + X} to reboot the processor.
\item Wait until the processors reboot information comes to the following lines:
\noindent
\begin{verbatim}
 Done executing startup script
 /home/sysuser/epics3.13.10/Application/iocBoot/iocbbc/stold.cmd
 bdb> checking system
\end{verbatim}

\begin{itemize}
 \item Please note that sometimes you might need to scroll up in case you missed this line appearing in the display.
 \item During the process the following GUIs will go white: upVPD, BBC LeCroy and ZDC. Ignore any messages you see on the Linux screens afterwards. If a GUI remains blank but the reset is complete, try to maximize and restore the GUI to their original size.
\end{itemize}

\item When the terminal is done updating and all GUIs are back, \textbf{turn on what needs to be on}: BBC HV, ZDC HV, EAST/WEST SMD and PP2PP HV. \textbf{Check the required detector state for the current conditions.}
\item Press \texttt{Ctrl + ]}. This should bring the prompt to \texttt{telnet>}.
\item Type \texttt{quit} and hit \texttt{[Enter]}.
\item Make an entry in the shift log when done.
\end{enumerate}

These instructions are (and should be) also placed in the STAR control room.

\subsection{Instructions for the TCIM reboot}
Sometimes, the ZDC RICH scalers suddenly change to weird values (like 200k for the ``ZDC and''). This can be caused by a fault in logic on the TCIM board and can be solved by rebooting it.
To do that, follow these steps:


\begin{enumerate}
\item Type into a terminal somewhere in the STAR network (e.g.\ from the \texttt{stargw} -- see section~\ref{computers})
\begin{verbatim}
ssh tof@tofrnps
\end{verbatim}
\item you will be asked for a password (it is in the password sheet)
\item enter
\item \verb=/off B4, y=
\item this turns off the TCIM. Wait 20--30 seconds
\item \verb=/on B4, y=
\item \verb=/x=
\end{enumerate}

\chapter{Tests of the hardware before the start of the collisions at RHIC}
The PMTs' response to cosmics needs to be checked before the run. After you turn the ZDC HV on, you can check the signal with an oscilloscope on the signal output at the patch panels close to the TCIM (See Fig.\ \ref{tcimpp}). A good way of doing this is to turn on the the HV on the towers one-by-one so you can see if the output of some ZDC towers is swapped.
Moreover, the ZDC cables may be swapped with the SMD, disconnected, or faulty. Therefore, if the signal is missing on the patch panel, you should follow the cables to the PMTs and check if all the cables are in the right place. Equally, you should check the HV connection.

\begin{figure}[htb]
\begin{center}
\includegraphics[width=.7\textwidth]{img/zdcepatch.jpg}
\end{center}
\caption{Outputs from PMTs at patch panel for ZDC east, that needs to be checked.}
\label{zdcepatch}
\end{figure}

ZDC west and east sum signals are typically checked on outputs of both TCIM and logic unit. ``ZDC and'' and ``ZDC kill'' signals are controlled on TCIM using  signal from a pulser. The pulser needs to be connected to one of the channels of ZDC East and West simultaneously. Typically, you divide the signal from the pulser and connect it to the ZDC East1 and West1 towers input of the TCIM. Now, you should monitor the RICH scalers output GUI on the \texttt{sc5} computer and the TCIM coincidence logic webpage at \url{172.16.15.101/AnalogandCoincidenceLogic13.html}.


\chapter{High-voltage controls}


\chapter{Run16 ZDC calibration\label{calibration}}

The voltages were calculated according to the formula
\begin{equation}
 G = aU^b
\end{equation}

where $G$ is a gain and $U$ stands for voltage. The coefficients $a$ and $b$ are 
extracted from \cite{ZDCvoltsDependence}
and are equal to approximately
\begin{equation}
b=4.2\,, \qquad a=4.0\,.
\end{equation}

We take into account that the desired position of the single neutron peak is at 60 ADC values
and the desired ratio between the towers is 6:3:1. To calculate the desired voltages we use the
formula
\begin{equation}
U_{\text{result}} = U_\text{current}\left(\frac{G_\text{desired}}{G} 
\frac{R_\text{desired}}{R}\right)^{1/4.2}
\end{equation}
where $G$ is the current position of the neutron peak, $G_\text{desired}$ is the desired position
of the neutron peak (currently 60), $R_{desired}$ is the desired ratio between the gain of the ADC SUM
tower and the current tower, and $R$ is the current ratio.
The resulting voltages are in Table \ref{uncorected}. The position of the single
neutron peaks in this test
are shown in Figures \ref{eastPlot} and \ref{westPlot}.
The ratios between the towers with the resulting voltages from the test
are shown in Table \ref{corected}.
Subsequently, new high-voltage values are also calculated.

\begin{table}[htb] 
\caption{Calculated voltages from the position of the single neutron peak and ratios 
between ZDC towers}
\label{uncorected}
\begin{center}
\begin{tabular}{lccccc}
 \toprule
 &$U_\text{current}$[V]&Single n pos.&gain ratio&desired gain ratio&$U_\text{result}$[V]\\
\midrule
 East&2471&38.33&0.689&0.6&2661\\
     &2779&38.33&0.228&0.3&3300\\
     &2353&38.33&0.083&0.1&2735\\
 \midrule
West&2532&43.56&0.665&0.6&2667\\
    &2643&43.56&0.278&0.3&2905\\
    &2671&43.56&0.057&0.1&3294\\
 \bottomrule
\end{tabular}
\end{center}
\end{table}

\begin{figure}[htb]
\begin{center}
\includegraphics[width=.8\textwidth]{img/CorrectedEast.pdf}
\end{center}
\caption{Single neutron peak for east towers. Resulting voltages from Table \ref{uncorected}.}
\label{eastPlot}
\end{figure}

\begin{figure}[htb]
\begin{center}
\includegraphics[width=.8\textwidth]{img/CorrectedWest.pdf}
\end{center}
\caption{Single neutron peak for west towers. Resulting voltages from Table \ref{uncorected}.}
\label{westPlot}
\end{figure}

\begin{table}[htb] 
\caption{Calculated voltages from the test with corrected high voltages.
The value 50 was set as the deseired position of the single neutron peak}
\label{corected}
\begin{center}
\begin{tabular}{lccccc}
 \toprule
 &$U_\text{current}$[V]&Single n pos.&gain ratio&desired gain ratio&$U_\text{result}$[V]\\
\midrule
 East&2547&53.32&0.569&0.6&2540\\
     &3159&53.32&0.330&0.3&3041\\
     &2618&53.32&0.101&0.1&2575\\
 \midrule
West&2333&31.07&0.563&0.6&2653\\
    &2542&31.07&0.319&0.3&2806\\
    &2883&31.07&0.118&0.1&3101\\
 \bottomrule
\end{tabular}
\end{center}
\end{table}

Channels where the voltage would exceed 3000 V in table \ref{corected} were set to 3000 V to
protect the PMTs. New single neutron peak as well as the double neutron peak were found and
plotted in Figures \ref{eastThird}
and \ref{westThird}. The position of the double neutron peak was set as twice the mean
of the single neutron peak.
The resulting voltages in the next iteration are in table \ref{thirdCalib}.

\begin{figure}[htb]
\begin{center}
\includegraphics[width=.8\textwidth]{img/neutronPeaksEastFinal.pdf}
\end{center}
\caption{Single neutron and double neutron peaks for east towers.
Voltages were taken from table~\ref{corected}.}
\label{eastThird}
\end{figure}

\begin{figure}[htb]
\begin{center}
\includegraphics[width=.8\textwidth]{img/neutronPeaksWestFinal.pdf}
\end{center}
\caption{Single neutron and double neutron peaks for west towers.
Voltages were taken from table~\ref{corected}.}
\label{westThird}
\end{figure}

\begin{table}[htb] 
\caption{Calculated voltages from the test with corrected high voltages from table~\ref{corected}.
The value 58 was set as the desired position of the single neutron peak}
\label{thirdCalib}
\begin{center}
\begin{tabular}{lccccc}
 \toprule
 &$U_\text{current}$[V]&Single n pos.&gain ratio&desired gain ratio&$U_\text{result}$[V]\\
\midrule
East  &2540  &58 &0.623  &0.6 &2517 \\
      &3000  &58 &0.274  &0.3 &3066 \\
      &2575  &58 &0.103	&0.1 &2557 \\
\midrule
West  &2653  &64 &0.634 &0.6 &2558 \\
      &2806  &64 &0.297 &0.3 &2748 \\
      &3000  &64 &0.069 &0.1 &3196 \\
\bottomrule
\end{tabular}
\end{center}
\end{table}

The results of the test with voltages from Table \ref{thirdCalib} are shown in
Figures \ref{oneMoreTestEast} and \ref{oneMoreTestWest}\@. The single neutron and double
neutron peaks were fitted. The ratios between the towers in the East were measured as
60.29:29.41:10.30 and in the West 61.07:30.52:8.41. 

In conclusion, the ratios between the towers were closer to the desired ones in the
test with voltages from Table \ref{thirdCalib}. The ratios between the single and double
neutron peak in the East towers were closer to the real values in the test from
Table \ref{corected}. Therefore, we suggest that the result voltages from Table \ref{corected}
are used for the East towers and the voltages from Table \ref{thirdCalib} are used for the
West towers. This setup was tested and the neutron peaks are plotted in Figures \ref{finalTestEast}
and \ref{finalTestWest}. The ratios between the towers were calculated as
61.30:28.58:10.17 for the East towers and 61.02:30.53:8.45 for the West towers.
Which is close enough to the desired values.

\begin{figure}[!htb]
\begin{center}
\includegraphics[width=.8\textwidth]{img/oneMoreTestEast.pdf}
\end{center}
\caption{Single neutron and double neutron peaks for east towers.
Voltages were taken from table~\ref{thirdCalib}.}
\label{oneMoreTestEast}
\end{figure}

\begin{figure}[!htb]
\begin{center}
\includegraphics[width=.8\textwidth]{img/oneMoreTestWest.pdf}
\end{center}
\caption{Single neutron and double neutron peaks for west towers. 
Voltages were taken from table~\ref{thirdCalib}.}
\label{oneMoreTestWest}
\end{figure}

\begin{figure}[!htb]
\begin{center}
\includegraphics[width=.8\textwidth]{img/FinalTestEast.pdf}
\end{center}
\caption{Single neutron and double neutron peaks for east towers in the final test. Voltages were
for the East towers taken from table~\ref{corected}.}
\label{finalTestEast}
\end{figure}

\begin{figure}[!htb]
\begin{center}
\includegraphics[width=.8\textwidth]{img/FinalTestWest.pdf}
\end{center}
\caption{Single neutron and double neutron peaks for west towersin the final test. Voltages
for the West towers were taken from table~\ref{thirdCalib}.}
\label{finalTestWest}
\end{figure}


\chapter{Running the ZDC calibration code}
\section{How to use the ZDC code}\hypertarget{how-to-use-the-zdc-code}{}\label{how-to-use-the-zdc-code}

All of the code is located in the folder: \texttt{/direct/star+u/msimko/ZDC}. You can copy the whole folder into
your directory (We will be calling it the base folder in this text). Some of the code has
a copy on Github as well~\cite{ZdcGithubRepo}.

\subsection{Getting the \verb=.dat=\ files}\hypertarget{getting-the-dat-files}{}\label{getting-the-dat-files}

First, you must download the \verb=.dat=\ files which are quite big, therefore We recommend saving them on
the \texttt{gpfs01} disc on RCF (\texttt{/gpfs/mnt/gpfs01/star/pwg}). For the data transfer, We use the script
\texttt{transfer.cpp} in\\
\texttt{/direct/star+u/msimko/ZDC/data}.\\
It uses Data Carousel so you won't put much
pressure on the HPSS (this is good practice). You run it by typing:
\begin{verbatim}
$ root -q transfer.cpp
\end{verbatim}
In this macro you have to change the variable
\texttt{runNumber} and where you want to put the files (variables: \texttt{Energy}, \texttt{trgSetup}, 
and \texttt{Workdir}).
Importantly, you must also change the number of \verb=.dat=\ files in the run. You can find it here:
\url{http://online.star.bnl.gov/RunLog/}~\cite{runLogBrowser}.

Now, you have to wait until your data get transfered. This can take from a couple of minutes up to a few hours.
You can monitor it here:
\url{https://www.star.bnl.gov/devcgi/display_accnt.cgi}~\cite{hpss}.
When they get transfered, running
\begin{verbatim}
$ ./ls.sh
\end{verbatim}
in the \texttt{data} folder will create a list of the transfered files.

\subsection{Making the TTree}\hypertarget{making-the-ttree}{}\label{making-the-ttree}

The analysis code is located in \texttt{StRoot} inside the base folder. First thing you need to do after you
copy the code is compiling it using
\begin{verbatim}
$ starver dev
$ cons
\end{verbatim}
To run the code, first you have to change a few things in the \texttt{config.C} macro in the base folder. 
Basically all of the
directories of the code and the data i.e. \texttt{dirOut}, \texttt{seenRuns}, and \texttt{fileName}. 
The \texttt{seenRuns} file has to
exist so you want to create one (\texttt{\$ touch seenRuns/seenRuns\_run16.msimko.txt}). If you want to use the code
on the same run multiple times (e.g.\ you screw up the first time), you have to delete the run number in
this file. You don't have to worry
about the rest.

You can run the code by typing:
\begin{verbatim}
$ starver dev
$ root4star -b -q ./read.C'("config.C")'
\end{verbatim}
or you can simply use the macro prepared for this:
\begin{verbatim}
$ ./read
\end{verbatim}
This will run the BFC chain (it can take a few minutes so this is a good time to have a coffee). When it is finished
in your \texttt{dirOut}, you will get lots of pdfs and pngs with histograms and, most importantly, a root file in
\texttt{dirOut/histo/someFileWithRunNumber.root}. Note that all of the histograms are also saved here.

\subsection{ZDC calibration code}\hypertarget{zdc-calibration-code}{}\label{zdc-calibration-code}



The TTree, you created in the previous step, stores the ADC readout value of ZDC modules, tof-multi, and
TDC values (timing) as well. 
So the next step of ZDC calibration
should be to read the information and plot figures.
We use some cuts in the code to extract the signal of the single neutron. In 2016, we used these cuts:

\begin{itemize}
\item TOF mult \textless{} 10
\item ZDC ADC sum in the oposite tower \textless{} 100
\item 200 \textless{} ZDC TDC (timing on both sides) \textless{} 2000
\end{itemize}

You can check the code in this repo in the files
\texttt{zdcTree.C} and \texttt{zdcTree.h}. Change the \texttt{runNumber}, \texttt{trgSetup},
and \texttt{typeEnergy} variables in \texttt{zdcTree.C}, and
the address of the input TFile in the file \texttt{zdcTree.h} (on lines 102 and 105). After this, you can run it by
typing
\begin{verbatim}
$ ./run.sh $tofMultCut
\end{verbatim}
The results should appear in the \texttt{dirOut/analysis/runNumber} folder. Now, you can fit the ZDC ADC distributions
and get the single and double neutron peaks (SNP and DNP, respectively). After you get the SNP, you can calibrate
the ZDC HV as described e.g.\ in the Chapter~\ref{calibration}.

\subsection{Creating an html page for monitoring}\hypertarget{creating-html-page-for-monitoring}{}\label{creating-html-page-for-monitoring}

You can also create a web page out of the results. All you have to do is running
\begin{verbatim}
$ root -l -q html_maker.cpp
\end{verbatim}
after you change variables \texttt{RunNumber}, \texttt{trgSetup}, \texttt{typeEnergy}, 
and \texttt{root\_file\_path} in the \texttt{html\_maker.cpp} file.



\begin{thebibliography}{9}
 \bibitem{ZDCvoltsDependence} Y.\ Zhu, \textit{ZDC Testing, Electronics Layout and Result Analysis}, 2010
 (cited February 7, 2017),
 \url{https://drupal.star.bnl.gov/STAR/system/files/ZDC_Document_20101005.pdf}
\end{thebibliography}

\end{document}

