\documentclass[a4paper,10pt]{report}
\usepackage[utf8]{inputenc}
\usepackage{amsmath}
\usepackage{amsfonts}
\usepackage{breakurl}
\usepackage{indentfirst}
\usepackage{graphicx}
\usepackage{enumerate}
\usepackage[breaklinks, hidelinks]{hyperref}
\usepackage{url}
\usepackage{booktabs}

\newcommand{\vect}[1]{\ensuremath{\boldsymbol{#1}}}
\newcommand{\ddd}{\ensuremath{\,\mathrm{d}}}
\newcommand{\dd}{\mathrm{d}}
\newcommand{\Var}{\mathrm{Var}}
\newcommand{\pravYi}{P(Y_i|\alpha, \beta, \epsilon_i)}
\newcommand{\pravY}{P(\boldsymbol{Y}|\alpha, \beta, \epsilon_i)}
\newcommand{\nsigma}{\frac{N}{\sigma^2}}
\newcommand{\arpxsq}{\overline{(X^2)}}
\newcommand{\sigman}{\frac{\sigma^2}{N}}
\newcommand{\jmenovatel}{ \arpxsq - \overline{X}^2}
\usepackage{amsmath}


%opening
\title{ZDC Operator's Manual}
\author{Miroslav Šimko, Lukáš Kramárik}
\date{February 2017}

\begin{document}

\maketitle

\tableofcontents
\chapter{Introduction}

\chapter{High-voltage controls}


\chapter{Run16 ZDC calibration\label{calibration}}

The voltages were calculated according to the formula
\begin{equation}
 G = aU^b
\end{equation}

where $G$ is a gain and $U$ stands for voltage. The coefficients $a$ and $b$ are 
extracted from \cite{ZDCvoltsDependence}
and are equal to approximately
\begin{equation}
b=4.2\,, \qquad a=4.0\,.
\end{equation}

We take into account that the desired position of the single neutron peak is at 60 ADC values
and the desired ratio between the towers is 6:3:1. To calculate the desired voltages we use the
formula
\begin{equation}
U_{\text{result}} = U_\text{current}\left(\frac{G_\text{desired}}{G} 
\frac{R_\text{desired}}{R}\right)^{1/4.2}
\end{equation}
where $G$ is the current position of the neutron peak, $G_\text{desired}$ is the desired position
of the neutron peak (currently 60), $R_{desired}$ is the desired ratio between the gain of the ADC SUM
tower and the current tower, and $R$ is the current ratio.
The resulting voltages are in Table \ref{uncorected}. The position of the single
neutron peaks in this test
are shown in Figures \ref{eastPlot} and \ref{westPlot}.
The ratios between the towers with the resulting voltages from the test
are shown in Table \ref{corected}.
Subsequently, new high-voltage values are also calculated.

\begin{table}[htb] 
\caption{Calculated voltages from the position of the single neutron peak and ratios 
between ZDC towers}
\label{uncorected}
\begin{center}
\begin{tabular}{lccccc}
 \toprule
 &$U_\text{current}$[V]&Single n pos.&gain ratio&desired gain ratio&$U_\text{result}$[V]\\
\midrule
 East&2471&38.33&0.689&0.6&2661\\
     &2779&38.33&0.228&0.3&3300\\
     &2353&38.33&0.083&0.1&2735\\
 \midrule
West&2532&43.56&0.665&0.6&2667\\
    &2643&43.56&0.278&0.3&2905\\
    &2671&43.56&0.057&0.1&3294\\
 \bottomrule
\end{tabular}
\end{center}
\end{table}

\begin{figure}[htb]
\begin{center}
\includegraphics[width=.8\textwidth]{img/CorrectedEast.pdf}
\end{center}
\caption{Single neutron peak for east towers. Resulting voltages from Table \ref{uncorected}.}
\label{eastPlot}
\end{figure}

\begin{figure}[htb]
\begin{center}
\includegraphics[width=.8\textwidth]{img/CorrectedWest.pdf}
\end{center}
\caption{Single neutron peak for west towers. Resulting voltages from Table \ref{uncorected}.}
\label{westPlot}
\end{figure}

\begin{table}[htb] 
\caption{Calculated voltages from the test with corrected high voltages.
The value 50 was set as the deseired position of the single neutron peak}
\label{corected}
\begin{center}
\begin{tabular}{lccccc}
 \toprule
 &$U_\text{current}$[V]&Single n pos.&gain ratio&desired gain ratio&$U_\text{result}$[V]\\
\midrule
 East&2547&53.32&0.569&0.6&2540\\
     &3159&53.32&0.330&0.3&3041\\
     &2618&53.32&0.101&0.1&2575\\
 \midrule
West&2333&31.07&0.563&0.6&2653\\
    &2542&31.07&0.319&0.3&2806\\
    &2883&31.07&0.118&0.1&3101\\
 \bottomrule
\end{tabular}
\end{center}
\end{table}

Channels where the voltage would exceed 3000 V in table \ref{corected} were set to 3000 V to
protect the PMTs. New single neutron peak as well as the double neutron peak were found and
plotted in Figures \ref{eastThird}
and \ref{westThird}. The position of the double neutron peak was set as twice the mean
of the single neutron peak.
The resulting voltages in the next iteration are in table \ref{thirdCalib}.

\begin{figure}[htb]
\begin{center}
\includegraphics[width=.8\textwidth]{img/neutronPeaksEastFinal.pdf}
\end{center}
\caption{Single neutron and double neutron peaks for east towers.
Voltages were taken from table~\ref{corected}.}
\label{eastThird}
\end{figure}

\begin{figure}[htb]
\begin{center}
\includegraphics[width=.8\textwidth]{img/neutronPeaksWestFinal.pdf}
\end{center}
\caption{Single neutron and double neutron peaks for west towers.
Voltages were taken from table~\ref{corected}.}
\label{westThird}
\end{figure}

\begin{table}[htb] 
\caption{Calculated voltages from the test with corrected high voltages from table~\ref{corected}.
The value 58 was set as the desired position of the single neutron peak}
\label{thirdCalib}
\begin{center}
\begin{tabular}{lccccc}
 \toprule
 &$U_\text{current}$[V]&Single n pos.&gain ratio&desired gain ratio&$U_\text{result}$[V]\\
\midrule
East  &2540  &58 &0.623  &0.6 &2517 \\
      &3000  &58 &0.274  &0.3 &3066 \\
      &2575  &58 &0.103	&0.1 &2557 \\
\midrule
West  &2653  &64 &0.634 &0.6 &2558 \\
      &2806  &64 &0.297 &0.3 &2748 \\
      &3000  &64 &0.069 &0.1 &3196 \\
\bottomrule
\end{tabular}
\end{center}
\end{table}

The results of the test with voltages from Table \ref{thirdCalib} are shown in
Figures \ref{oneMoreTestEast} and \ref{oneMoreTestWest}\@. The single neutron and double
neutron peaks were fitted. The ratios between the towers in the East were measured as
60.29:29.41:10.30 and in the West 61.07:30.52:8.41. 

In conclusion, the ratios between the towers were closer to the desired ones in the
test with voltages from Table \ref{thirdCalib}. The ratios between the single and double
neutron peak in the East towers were closer to the real values in the test from
Table \ref{corected}. Therefore, we suggest that the result voltages from Table \ref{corected}
are used for the East towers and the voltages from Table \ref{thirdCalib} are used for the
West towers. This setup was tested and the neutron peaks are plotted in Figures \ref{finalTestEast}
and \ref{finalTestWest}. The ratios between the towers were calculated as
61.30:28.58:10.17 for the East towers and 61.02:30.53:8.45 for the West towers.
Which is close enough to the desired values.

\begin{figure}[!htb]
\begin{center}
\includegraphics[width=.8\textwidth]{img/oneMoreTestEast.pdf}
\end{center}
\caption{Single neutron and double neutron peaks for east towers.
Voltages were taken from table~\ref{thirdCalib}.}
\label{oneMoreTestEast}
\end{figure}

\begin{figure}[!htb]
\begin{center}
\includegraphics[width=.8\textwidth]{img/oneMoreTestWest.pdf}
\end{center}
\caption{Single neutron and double neutron peaks for west towers. 
Voltages were taken from table~\ref{thirdCalib}.}
\label{oneMoreTestWest}
\end{figure}

\begin{figure}[!htb]
\begin{center}
\includegraphics[width=.8\textwidth]{img/FinalTestEast.pdf}
\end{center}
\caption{Single neutron and double neutron peaks for east towers in the final test. Voltages were
for the East towers taken from table~\ref{corected}.}
\label{finalTestEast}
\end{figure}

\begin{figure}[!htb]
\begin{center}
\includegraphics[width=.8\textwidth]{img/FinalTestWest.pdf}
\end{center}
\caption{Single neutron and double neutron peaks for west towersin the final test. Voltages
for the West towers were taken from table~\ref{thirdCalib}.}
\label{finalTestWest}
\end{figure}


\chapter{Running the ZDC calibration code}
\section{How to use the ZDC code}\hypertarget{how-to-use-the-zdc-code}{}\label{how-to-use-the-zdc-code}

All of the code is located in the folder: \texttt{/direct/star+u/msimko/ZDC}. You can copy the whole folder into
your directory (We will be calling it the base folder in this text). Some of the code has
a copy on Github as well~\cite{ZdcGithubRepo}.

\subsection{Getting the \verb=.dat=\ files}\hypertarget{getting-the-dat-files}{}\label{getting-the-dat-files}

First, you must download the \verb=.dat=\ files which are quite big, therefore We recommend saving them on
the \texttt{gpfs01} disc on RCF (\texttt{/gpfs/mnt/gpfs01/star/pwg}). For the data transfer, We use the script
\texttt{transfer.cpp} in\\
\texttt{/direct/star+u/msimko/ZDC/data}.\\
It uses Data Carousel so you won't put much
pressure on the HPSS (this is good practice). You run it by typing:
\begin{verbatim}
$ root -q transfer.cpp
\end{verbatim}
In this macro you have to change the variable
\texttt{runNumber} and where you want to put the files (variables: \texttt{Energy}, \texttt{trgSetup}, 
and \texttt{Workdir}).
Importantly, you must also change the number of \verb=.dat=\ files in the run. You can find it here:
\url{http://online.star.bnl.gov/RunLog/}~\cite{runLogBrowser}.

Now, you have to wait until your data get transfered. This can take from a couple of minutes up to a few hours.
You can monitor it here:
\url{https://www.star.bnl.gov/devcgi/display_accnt.cgi}~\cite{hpss}.
When they get transfered, running
\begin{verbatim}
$ ./ls.sh
\end{verbatim}
in the \texttt{data} folder will create a list of the transfered files.

\subsection{Making the TTree}\hypertarget{making-the-ttree}{}\label{making-the-ttree}

The analysis code is located in \texttt{StRoot} inside the base folder. First thing you need to do after you
copy the code is compiling it using
\begin{verbatim}
$ starver dev
$ cons
\end{verbatim}
To run the code, first you have to change a few things in the \texttt{config.C} macro in the base folder. 
Basically all of the
directories of the code and the data i.e. \texttt{dirOut}, \texttt{seenRuns}, and \texttt{fileName}. 
The \texttt{seenRuns} file has to
exist so you want to create one (\texttt{\$ touch seenRuns/seenRuns\_run16.msimko.txt}). If you want to use the code
on the same run multiple times (e.g.\ you screw up the first time), you have to delete the run number in
this file. You don't have to worry
about the rest.

You can run the code by typing:
\begin{verbatim}
$ starver dev
$ root4star -b -q ./read.C'("config.C")'
\end{verbatim}
or you can simply use the macro prepared for this:
\begin{verbatim}
$ ./read
\end{verbatim}
This will run the BFC chain (it can take a few minutes so this is a good time to have a coffee). When it is finished
in your \texttt{dirOut}, you will get lots of pdfs and pngs with histograms and, most importantly, a root file in
\texttt{dirOut/histo/someFileWithRunNumber.root}. Note that all of the histograms are also saved here.

\subsection{ZDC calibration code}\hypertarget{zdc-calibration-code}{}\label{zdc-calibration-code}



The TTree, you created in the previous step, stores the ADC readout value of ZDC modules, tof-multi, and
TDC values (timing) as well. 
So the next step of ZDC calibration
should be to read the information and plot figures.
We use some cuts in the code to extract the signal of the single neutron. In 2016, we used these cuts:

\begin{itemize}
\item TOF mult \textless{} 10
\item ZDC ADC sum in the oposite tower \textless{} 100
\item 200 \textless{} ZDC TDC (timing on both sides) \textless{} 2000
\end{itemize}

You can check the code in this repo in the files
\texttt{zdcTree.C} and \texttt{zdcTree.h}. Change the \texttt{runNumber}, \texttt{trgSetup},
and \texttt{typeEnergy} variables in \texttt{zdcTree.C}, and
the address of the input TFile in the file \texttt{zdcTree.h} (on lines 102 and 105). After this, you can run it by
typing
\begin{verbatim}
$ ./run.sh $tofMultCut
\end{verbatim}
The results should appear in the \texttt{dirOut/analysis/runNumber} folder. Now, you can fit the ZDC ADC distributions
and get the single and double neutron peaks (SNP and DNP, respectively). After you get the SNP, you can calibrate
the ZDC HV as described e.g.\ in the Chapter~\ref{calibration}.

\subsection{Creating an html page for monitoring}\hypertarget{creating-html-page-for-monitoring}{}\label{creating-html-page-for-monitoring}

You can also create a web page out of the results. All you have to do is running
\begin{verbatim}
$ root -l -q html_maker.cpp
\end{verbatim}
after you change variables \texttt{RunNumber}, \texttt{trgSetup}, \texttt{typeEnergy}, 
and \texttt{root\_file\_path} in the \texttt{html\_maker.cpp} file.



\begin{thebibliography}{9}
 \bibitem{ZDCvoltsDependence} Y.\ Zhu, \textit{ZDC Testing, Electronics Layout and Result Analysis}, 2010
 (cited February 7, 2017),
 \url{https://drupal.star.bnl.gov/STAR/system/files/ZDC_Document_20101005.pdf}
\end{thebibliography}

\end{document}

